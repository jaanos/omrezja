\documentclass{article}
\usepackage[utf8]{inputenc}
\usepackage[oznake]{omrezja}

\pgfrealjobname{vzorec}

\title{Omrežja za optimizacijske naloge}
\author{Janoš Vidali}

\begin{document}

\maketitle

\section{Problem najcenejšega razvoza}

Definirajmo in izrišimo omrežje za problem razvoza.

\NewEnviron{razvoz@grafA}{
    \vozlisce{a}[left]{-8, 5}[1]
    \vozlisce{b}[right]{ 8, 5}[5]
    \vozlisce{c}[right]{ 8,-5}[-4]
    \vozlisce{d}[left]{-8,-5}[-2]
    \vozlisce{e}[above]{-3, 3}[10]
    \vozlisce{f}[above]{ 3, 3}[-7]
    \vozlisce{g}[below]{ 0,-2}[-3]

    \povezava{a}{b}{3}{above}
    \povezava{c}{b}{9}{right}
    \povezava{c}{d}{4}{below}
    \povezava{d}{a}{1}{left}
    \povezava{e}{a}{2}[sloped]{above}
    \povezava{e}{d}{11}[sloped]{above}
    \povezava{f}{b}{4}[sloped]{above}
    \povezava{f}{e}{8}{above}
    \povezava{g}{c}{2}[sloped]{below}
    \povezava{g}{d}{17}[sloped]{below}
    \povezava{g}{e}{1}[sloped]{above}
    \povezava{g}{f}{9}[sloped]{above}
}

\begin{razvoz}{grafA}
\end{razvoz}

\noindent
Omrežje lahko dopolnimo z novimi vozlišči ali povezavami
ter po potrebi spremenimo cene povezav.

\NewEnviron{razvoz@grafAs}{
    \cena{a}{b}{0}
    \cena{c}{b}{0}
    \cena{c}{d}{0}
    \cena{d}{a}{0}
    \cena{e}{a}{0}
    \cena{e}{d}{0}
    \cena{f}{b}{0}
    \cena{f}{e}{0}
    \cena{g}{c}{0}
    \cena{g}{d}{0}
    \cena{g}{e}{0}
    \cena{g}{f}{0}

    \begin{razvoz@grafA}
    \end{razvoz@grafA}

    \povezava{c}{g}{1}[sloped]{above}[bend right=15]
    \povezava{d}{g}{1}[sloped]{above}[bend left=15]
    \povezava{f}{g}{1}[sloped]{below}[bend left=15]
    \povezava{g}{a}{1}[sloped]{below}
    \povezava{g}{b}{1}[sloped]{below}[bend right=20]
}

\clearpage

\noindent
Poiščimo drevesno dopustno rešitev,
označimo kandidate za vstop,
ter izberimo vstopno in izstopno povezavo.

\begin{razvoz}{grafAs}
    \kolicina{c}{g}{4}[izstopi]
    \kolicina{d}{g}{2}
    \kolicina{f}{g}{7}
    \kolicina{g}{a}{1}
    \kolicina{g}{b}{5}
    \kolicina{g}{e}{10}

    \oznaka{g}{0}
    \oznaka{c}{-1}
    \oznaka{d}{-1}
    \oznaka{f}{-1}
    \oznaka{a}{1}
    \oznaka{b}{1}
    \oznaka{e}{0}

    \kandidat{c}{b}[vstopi]
    \kandidat{d}{a}
    \kandidat{e}{a}
    \kandidat{f}{b}
    \kandidat{f}{e}
\end{razvoz}

\clearpage

\section{Problem maksimalnega pretoka in minimalnega prereza}

Definirajmo in izrišimo omrežje za problem pretoka.

\NewEnviron{pretok@grafB}{
    \vozlisce*{S}{-8, 5}
    \vozlisce{a}{ 0, 5}
    \vozlisce{b}{-4, 2}
    \vozlisce{c}{-8,-1}
    \vozlisce{d}{ 0,-1}
    \vozlisce{e}{ 4, 2}
    \vozlisce{f}{ 0,-5}
    \vozlisce{g}{ 8, 5}
    \vozlisce*{T}{ 8,-1}

    \povezava{S}{a}{6}{above}
    \povezava{S}{b}{3}[sloped]{above}
    \povezava{S}{c}{2}{left}
    \povezava{a}{b}{2}[sloped]{above}
    \povezava{a}{e}{9}[sloped]{above}
    \povezava{b}{c}{7}[sloped]{below}
    \povezava{c}{d}{5}{below}
    \povezava{c}{f}{1}[sloped]{below}
    \povezava{d}{a}{10}{left}
    \povezava{d}{b}{10}[sloped]{below}
    \povezava{d}{e}{3}[sloped]{below}
    \povezava{d}{T}{2}{below}
    \povezava{e}{g}{2}[sloped]{above}
    \povezava{e}{T}{5}[sloped]{below}
    \povezava{f}{d}{8}{left}
    \povezava{f}{T}{4}[sloped]{below}
    \povezava{g}{a}{3}{above}
    \povezava{g}{T}{4}{right}
}

\begin{pretok}{grafB}
\end{pretok}

\noindent
Poiščimo disjunktne povečujoče poti.

\begin{pretok}{grafB}
    \nicle
    \kolicina*{S}{a}[6]{0}[red]
    \kolicina*{a}{e}[9]{0}[red]
    \kolicina*{e}{T}[5]{0}[red]
    \kolicina*{S}{b}[3]{0}[green]
    \kolicina*{b}{c}[7]{0}[green]
    \kolicina*{c}{d}[5]{0}[green]
    \kolicina*{d}{T}[2]{0}[green]
    \kolicina*{S}{c}[2]{0}[blue]
    \kolicina*{c}{f}[1]{0}[blue]
    \kolicina*{f}{T}[4]{0}[blue]
\end{pretok}

\noindent
Podajmo maksimalni pretok in minimalni prerez.

\begin{pretok}{grafB}
    \nicle
    \kolicina{S}{a}[0]{6}[zasicena]
    \kolicina{S}{b}[1]{2}
    \kolicina{S}{c}[0]{2}[zasicena]
    \kolicina{a}{e}[3]{6}
    \kolicina{b}{c}[5]{2}
    \kolicina{c}{d}[2]{3}
    \kolicina{c}{f}[0]{1}[prerez]
    \kolicina{d}{e}[2]{1}
    \kolicina{d}{T}[0]{2}[prerez]
    \kolicina{e}{g}[0]{2}[prerez]
    \kolicina{e}{T}[0]{5}[prerez]
    \kolicina{f}{T}[3]{1}
    \kolicina{g}{T}[2]{2}

    \nasprotna{f}{d}
    \nasprotna{g}{a}

    \dosezeno{S}
    \dosezeno{b}
    \dosezeno{c}
    \dosezeno{d}
    \dosezeno{a}
    \dosezeno{e}
\end{pretok}

\end{document}
