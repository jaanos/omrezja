\documentclass{article}
\usepackage[utf8]{inputenc}
\usepackage{amsmath}
\usepackage[oznake]{omrezja}

\pgfrealjobname{vzorec}

\title{Omrežja za optimizacijske naloge}
\author{Janoš Vidali}

\begin{document}

\maketitle

\section{Problem najcenejšega razvoza}

Definirajmo in izrišimo omrežje za problem razvoza.

\NewEnviron{razvoz@grafA}{
    \vozlisce{a}[left]{-8, 5}[1]
    \vozlisce{b}[right]{ 8, 5}[5]
    \vozlisce{c}[right]{ 8,-5}[-4]
    \vozlisce{d}[left]{-8,-5}[-2]
    \vozlisce{e}[above]{-3, 3}[10]
    \vozlisce{f}[above]{ 3, 3}[-7]
    \vozlisce{g}[below]{ 0,-2}[-3]

    \povezava{a}{b}{3}{above}
    \povezava{c}{b}{9}{right}
    \povezava{c}{d}{4}{below}
    \povezava{d}{a}{1}{left}
    \povezava{e}{a}{2}[sloped]{above}
    \povezava{e}{d}{11}[sloped]{above}
    \povezava{f}{b}{4}[sloped]{above}
    \povezava{f}{e}{8}{above}
    \povezava{g}{c}{2}[sloped]{below}
    \povezava{g}{d}{17}[sloped]{below}
    \povezava{g}{e}{1}[sloped]{above}
    \povezava{g}{f}{9}[sloped]{above}
}

\begin{razvoz}{grafA}
\end{razvoz}

\noindent
Omrežje lahko dopolnimo z novimi vozlišči ali povezavami
ter po potrebi spremenimo cene povezav.

\NewEnviron{razvoz@grafAs}{
    \cena{a}{b}{0}
    \cena{c}{b}{0}
    \cena{c}{d}{0}
    \cena{d}{a}{0}
    \cena{e}{a}{0}
    \cena{e}{d}{0}
    \cena{f}{b}{0}
    \cena{f}{e}{0}
    \cena{g}{c}{0}
    \cena{g}{d}{0}
    \cena{g}{e}{0}
    \cena{g}{f}{0}

    \begin{razvoz@grafA}
    \end{razvoz@grafA}

    \povezava{c}{g}{1}[sloped]{above}[bend right=15]
    \povezava{d}{g}{1}[sloped]{above}[bend left=15]
    \povezava{f}{g}{1}[sloped]{below}[bend left=15]
    \povezava{g}{a}{1}[sloped]{below}
    \povezava{g}{b}{1}[sloped]{below}[bend right=20]
}

\clearpage

\noindent
Poiščimo drevesno dopustno rešitev,
označimo kandidate za vstop,
ter izberimo vstopno in izstopno povezavo.

\begin{razvoz}{grafAs}
    \kolicina{c}{g}{4}[izstopi]
    \kolicina{d}{g}{2}
    \kolicina{f}{g}{7}
    \kolicina{g}{a}{1}
    \kolicina{g}{b}{5}
    \kolicina{g}{e}{10}

    \oznaka{g}{0}
    \oznaka{c}{-1}
    \oznaka{d}{-1}
    \oznaka{f}{-1}
    \oznaka{a}{1}
    \oznaka{b}{1}
    \oznaka{e}{0}

    \kandidat{c}{b}[vstopi]
    \kandidat{d}{a}
    \kandidat{e}{a}
    \kandidat{f}{b}
    \kandidat{f}{e}
\end{razvoz}

\makebox[\textwidth][c]{
\begin{mreza}{2}{3}
    \begin{razvoz*}{grafA}
    \end{razvoz*}
\end{mreza}
}

\clearpage

\section{Problem maksimalnega pretoka in minimalnega prereza}

Definirajmo in izrišimo omrežje za problem pretoka.

\NewEnviron{pretok@grafB}{
    \vozlisce*{s}{-8, 5}
    \vozlisce{a}{ 0, 5}
    \vozlisce{b}{-4, 2}
    \vozlisce{c}{-8,-1}
    \vozlisce{d}{ 0,-1}
    \vozlisce{e}{ 4, 2}
    \vozlisce{f}{ 0,-5}
    \vozlisce{g}{ 8, 5}
    \vozlisce*{t}{ 8,-1}

    \povezava{s}{a}{6}{above}
    \povezava{s}{b}{3}[sloped]{above}
    \povezava{s}{c}{2}{left}
    \povezava{a}{b}{2}[sloped]{above}
    \povezava{a}{e}{9}[sloped]{above}
    \povezava{b}{c}{7}[sloped]{below}
    \povezava{c}{d}{5}{below}
    \povezava{c}{f}{1}[sloped]{below}
    \povezava{d}{a}{10}{left}
    \povezava{d}{b}{10}[sloped]{below}
    \povezava{d}{e}{3}[sloped]{below}
    \povezava{d}{t}{2}{below}
    \povezava{e}{g}{2}[sloped]{above}
    \povezava{e}{t}{5}[sloped]{below}
    \povezava{f}{d}{8}{left}
    \povezava{f}{t}{4}[sloped]{below}
    \povezava{g}{a}{3}{above}
    \povezava{g}{t}{4}{right}
}

\begin{pretok}{grafB}
\end{pretok}

\noindent
Poiščimo disjunktne povečujoče poti.

\begin{pretok}{grafB}
    \nicle
    \kolicina*{s}{a}[6]{0}[red]
    \kolicina*{a}{e}[9]{0}[red]
    \kolicina*{e}{t}[5]{0}[red]
    \kolicina*{s}{b}[3]{0}[green]
    \kolicina*{b}{c}[7]{0}[green]
    \kolicina*{c}{d}[5]{0}[green]
    \kolicina*{d}{t}[2]{0}[green]
    \kolicina*{s}{c}[2]{0}[blue]
    \kolicina*{c}{f}[1]{0}[blue]
    \kolicina*{f}{t}[4]{0}[blue]

    \dosezeno{s}
    \dosezeno{a}
    \dosezeno{b}
    \dosezeno{c}
    \dosezeno{d}
    \dosezeno{e}
    \dosezeno{f}
    \dosezeno{g}
    \dosezeno{t}[pogojno]
\end{pretok}

\noindent
Podajmo maksimalni pretok in minimalni prerez.

\begin{pretok}{grafB}
    \nicle
    \kolicina{s}{a}[0]{6}[zasicena]
    \kolicina{s}{b}[1]{2}
    \kolicina{s}{c}[0]{2}[zasicena]
    \kolicina{a}{e}[3]{6}
    \kolicina{b}{c}[5]{2}
    \kolicina{c}{d}[2]{3}
    \kolicina{c}{f}[0]{1}[prerez]
    \kolicina{d}{e}[2]{1}
    \kolicina{d}{t}[0]{2}[prerez]
    \kolicina{e}{g}[0]{2}[prerez]
    \kolicina{e}{t}[0]{5}[prerez]
    \kolicina{f}{t}[3]{1}
    \kolicina{g}{t}[2]{2}

    \nasprotna{f}{d}
    \nasprotna{g}{a}

    \dosezeno{s}
    \dosezeno{b}
    \dosezeno{c}
    \dosezeno{d}
    \dosezeno{a}
    \dosezeno{e}
\end{pretok}

\makebox[\textwidth][c]{
\begin{mreza}{2}{3}
    \begin{pretok*}{grafB}
    \end{pretok*}
\end{mreza}
}

\clearpage

\section{Neusmerjeni grafi}

Narišimo neusmerjen graf.

\NewEnviron{neusmerjen@grafC}{
    \vozlisce{a}[above]{xyz polar cs:angle=  0,radius=0}
    \vozlisce{b}[below]{xyz polar cs:angle=270,radius=3}
    \vozlisce{c}[right]{xyz polar cs:angle= 30,radius=3}
    \vozlisce{d}[left]{xyz polar cs:angle=150,radius=3}

    \povezava{a}{b}{}{left}
    \povezava{a}{c}{}[sloped]{below}
    \povezava{a}{d}{}[sloped]{below}
    \povezava{b}{c}{}[sloped]{below}
    \povezava{b}{d}{}[sloped]{below}
    \povezava{c}{d}{}{above}
}

\begin{neusmerjen}{grafC}
\end{neusmerjen}

\noindent
Dodajmo uteži na vozlišča in povezave.

\begin{neusmerjen}{grafC}
    \oznaka{a}{0}
    \oznaka{b}{1}
    \oznaka{c}{2}
    \oznaka{d}{3}

    \kolicina{a}{b}{4}
    \kolicina{a}{c}{5}
    \kolicina{a}{d}{6}
    \kolicina{b}{c}{7}
    \kolicina{b}{d}{8}
    \kolicina{c}{d}{9}
\end{neusmerjen}

\makebox[\textwidth][c]{
    \begin{mreza}{3}{4}
        \begin{neusmerjen*}{grafC}
        \end{neusmerjen*}
    \end{mreza}
}

\clearpage

\section{Usmerjeni grafi}

Narišimo usmerjen graf.

\NewEnviron{usmerjen@grafD}{
    \vozlisce{s}{-7, 0}
    \vozlisce{a}{-3, 0}
    \vozlisce{b}{-5, 2}
    \vozlisce{c}{-5, 0}
    \vozlisce{d}{ 1,-2}
    \vozlisce{e}{-1, 0}
    \vozlisce{f}{ 1, 0}
    \vozlisce{g}{ 3, 0}
    \vozlisce{h}{ 3, 2}
    \vozlisce{i}{ 5, 0}
    \vozlisce{j}{-5,-2}
    \vozlisce{t}{ 7, 0}

    \povezava{s}{b}{0}{above}[sloped]
    \povezava*{s}{c}{0}{above}
    \povezava{s}{j}{0}{below}[sloped]
    \povezava*{a}{e}{5}{above}
    \povezava{b}{h}{3}{above}
    \povezava*{c}{a}{5}{below}
    \povezava{d}{t}{2}{below}[sloped]
    \povezava{e}{d}{5}{below, near end}[sloped]
    \povezava*{e}{f}{5}{below}
    \povezava{f}{g}{2}{above}
    \povezava*{f}{h}{2}{above}[sloped]
    \povezava{g}{i}{1}{below}
    \povezava*{h}{i}{3}{above}[sloped]
    \povezava*{i}{t}{8}{above}
    \povezava{j}{g}{1}{below, near start}[sloped]
}

\begin{usmerjen}[scale=1.5]{grafD}
\end{usmerjen}

\makebox[\textwidth][c]{
    \begin{mreza}{4}{3}
        \begin{usmerjen*}{grafD}
        \end{usmerjen*}
    \end{mreza}
}

\clearpage

\section{Odločitvena drevesa}

Narišimo odločitveno drevo.

\begin{drevo}[scale=1.5]*
    \setboolean{@oznake}{false}

    \vozlisce[odlocitev]{A}[right]{-8,-1}[27]
    \vozlisce[dogodek]{B}[above]{-4, 2}[26]
    \vozlisce[odlocitev]{C}[below]{-4,-4}[27]
    \vozlisce[odlocitev]{D}[above]{ 0, 4}[35]
    \vozlisce[odlocitev]{E}[below]{ 0, 0}[17]
    \vozlisce[dogodek]{F}[above]{ 0,-3}[27]
    \vozlisce{G}{ 1,-5}[15]
    \vozlisce[dogodek]{H}[above]{ 4, 5}[35]
    \vozlisce{I}{ 5, 3}[14]
    \vozlisce[dogodek]{J}[above]{ 4, 1}[17]
    \vozlisce{K}{ 5,-1}[14]
    \vozlisce{L}{ 4,-2}[54]
    \vozlisce{M}{ 4,-4}[0 ]
    \vozlisce{N}{ 8, 6}[53]
    \vozlisce{O}{ 8, 4}[-1]
    \vozlisce{P}{ 8, 2}[53]
    \vozlisce{Q}{ 8, 0}[-1]

    \povezava{A}{B}{\text{naročimo raziskavo}}[sloped]{above}
    \povezava*{A}{C}{\text{ne naročimo raziskave}}[sloped]{below}
    \povezava{B}{D}{\text{pozitivna: $1/2$}}[sloped]{above}
    \povezava{B}{E}{\text{negativna: $1/2$}}[sloped]{below}
    \povezava*{C}{F}{\text{sami prodajamo}}[sloped]{above}
    \povezava{C}{G}{\text{prodamo konkurenci}}[sloped]{below}
    \povezava*{D}{H}{\text{sami prodajamo}}[sloped]{above}
    \povezava{D}{I}{\text{prodamo konkurenci}}[sloped]{below}
    \povezava*{E}{J}{\text{sami prodajamo}}[sloped]{above}
    \povezava{E}{K}{\text{prodamo konkurenci}}[sloped]{below}
    \povezava{F}{L}{\text{uspeh: $1/2$}}[sloped]{above}
    \povezava{F}{M}{\text{neuspeh: $1/2$}}[sloped]{below}
    \povezava{H}{N}{\text{uspeh: $2/3$}}[sloped]{above}
    \povezava{H}{O}{\text{neuspeh: $1/3$}}[sloped]{below}
    \povezava{J}{P}{\text{uspeh: $1/3$}}[sloped]{above}
    \povezava{J}{Q}{\text{neuspeh: $2/3$}}[sloped]{below}
\end{drevo}

\end{document}
